
% Default to the notebook output style

    


% Inherit from the specified cell style.




    
\documentclass{article}

    
    
    \usepackage{graphicx} % Used to insert images
    \usepackage{adjustbox} % Used to constrain images to a maximum size 
    \usepackage{color} % Allow colors to be defined
    \usepackage{enumerate} % Needed for markdown enumerations to work
    \usepackage{geometry} % Used to adjust the document margins
    \usepackage{amsmath} % Equations
    \usepackage{amssymb} % Equations
    \usepackage{eurosym} % defines \euro
    \usepackage[mathletters]{ucs} % Extended unicode (utf-8) support
    \usepackage[utf8x]{inputenc} % Allow utf-8 characters in the tex document
    \usepackage{fancyvrb} % verbatim replacement that allows latex
    \usepackage{grffile} % extends the file name processing of package graphics 
                         % to support a larger range 
    % The hyperref package gives us a pdf with properly built
    % internal navigation ('pdf bookmarks' for the table of contents,
    % internal cross-reference links, web links for URLs, etc.)
    \usepackage{hyperref}
    \usepackage{longtable} % longtable support required by pandoc >1.10
    \usepackage{booktabs}  % table support for pandoc > 1.12.2
    

    
    
    \definecolor{orange}{cmyk}{0,0.4,0.8,0.2}
    \definecolor{darkorange}{rgb}{.71,0.21,0.01}
    \definecolor{darkgreen}{rgb}{.12,.54,.11}
    \definecolor{myteal}{rgb}{.26, .44, .56}
    \definecolor{gray}{gray}{0.45}
    \definecolor{lightgray}{gray}{.95}
    \definecolor{mediumgray}{gray}{.8}
    \definecolor{inputbackground}{rgb}{.95, .95, .85}
    \definecolor{outputbackground}{rgb}{.95, .95, .95}
    \definecolor{traceback}{rgb}{1, .95, .95}
    % ansi colors
    \definecolor{red}{rgb}{.6,0,0}
    \definecolor{green}{rgb}{0,.65,0}
    \definecolor{brown}{rgb}{0.6,0.6,0}
    \definecolor{blue}{rgb}{0,.145,.698}
    \definecolor{purple}{rgb}{.698,.145,.698}
    \definecolor{cyan}{rgb}{0,.698,.698}
    \definecolor{lightgray}{gray}{0.5}
    
    % bright ansi colors
    \definecolor{darkgray}{gray}{0.25}
    \definecolor{lightred}{rgb}{1.0,0.39,0.28}
    \definecolor{lightgreen}{rgb}{0.48,0.99,0.0}
    \definecolor{lightblue}{rgb}{0.53,0.81,0.92}
    \definecolor{lightpurple}{rgb}{0.87,0.63,0.87}
    \definecolor{lightcyan}{rgb}{0.5,1.0,0.83}
    
    % commands and environments needed by pandoc snippets
    % extracted from the output of `pandoc -s`
    \providecommand{\tightlist}{%
      \setlength{\itemsep}{0pt}\setlength{\parskip}{0pt}}
    \DefineVerbatimEnvironment{Highlighting}{Verbatim}{commandchars=\\\{\}}
    % Add ',fontsize=\small' for more characters per line
    \newenvironment{Shaded}{}{}
    \newcommand{\KeywordTok}[1]{\textcolor[rgb]{0.00,0.44,0.13}{\textbf{{#1}}}}
    \newcommand{\DataTypeTok}[1]{\textcolor[rgb]{0.56,0.13,0.00}{{#1}}}
    \newcommand{\DecValTok}[1]{\textcolor[rgb]{0.25,0.63,0.44}{{#1}}}
    \newcommand{\BaseNTok}[1]{\textcolor[rgb]{0.25,0.63,0.44}{{#1}}}
    \newcommand{\FloatTok}[1]{\textcolor[rgb]{0.25,0.63,0.44}{{#1}}}
    \newcommand{\CharTok}[1]{\textcolor[rgb]{0.25,0.44,0.63}{{#1}}}
    \newcommand{\StringTok}[1]{\textcolor[rgb]{0.25,0.44,0.63}{{#1}}}
    \newcommand{\CommentTok}[1]{\textcolor[rgb]{0.38,0.63,0.69}{\textit{{#1}}}}
    \newcommand{\OtherTok}[1]{\textcolor[rgb]{0.00,0.44,0.13}{{#1}}}
    \newcommand{\AlertTok}[1]{\textcolor[rgb]{1.00,0.00,0.00}{\textbf{{#1}}}}
    \newcommand{\FunctionTok}[1]{\textcolor[rgb]{0.02,0.16,0.49}{{#1}}}
    \newcommand{\RegionMarkerTok}[1]{{#1}}
    \newcommand{\ErrorTok}[1]{\textcolor[rgb]{1.00,0.00,0.00}{\textbf{{#1}}}}
    \newcommand{\NormalTok}[1]{{#1}}
    
    % Define a nice break command that doesn't care if a line doesn't already
    % exist.
    \def\br{\hspace*{\fill} \\* }
    % Math Jax compatability definitions
    \def\gt{>}
    \def\lt{<}
    % Document parameters
    \title{Summaries}
    
    
    

    % Pygments definitions
    
\makeatletter
\def\PY@reset{\let\PY@it=\relax \let\PY@bf=\relax%
    \let\PY@ul=\relax \let\PY@tc=\relax%
    \let\PY@bc=\relax \let\PY@ff=\relax}
\def\PY@tok#1{\csname PY@tok@#1\endcsname}
\def\PY@toks#1+{\ifx\relax#1\empty\else%
    \PY@tok{#1}\expandafter\PY@toks\fi}
\def\PY@do#1{\PY@bc{\PY@tc{\PY@ul{%
    \PY@it{\PY@bf{\PY@ff{#1}}}}}}}
\def\PY#1#2{\PY@reset\PY@toks#1+\relax+\PY@do{#2}}

\expandafter\def\csname PY@tok@gd\endcsname{\def\PY@tc##1{\textcolor[rgb]{0.63,0.00,0.00}{##1}}}
\expandafter\def\csname PY@tok@gu\endcsname{\let\PY@bf=\textbf\def\PY@tc##1{\textcolor[rgb]{0.50,0.00,0.50}{##1}}}
\expandafter\def\csname PY@tok@gt\endcsname{\def\PY@tc##1{\textcolor[rgb]{0.00,0.27,0.87}{##1}}}
\expandafter\def\csname PY@tok@gs\endcsname{\let\PY@bf=\textbf}
\expandafter\def\csname PY@tok@gr\endcsname{\def\PY@tc##1{\textcolor[rgb]{1.00,0.00,0.00}{##1}}}
\expandafter\def\csname PY@tok@cm\endcsname{\let\PY@it=\textit\def\PY@tc##1{\textcolor[rgb]{0.25,0.50,0.50}{##1}}}
\expandafter\def\csname PY@tok@vg\endcsname{\def\PY@tc##1{\textcolor[rgb]{0.10,0.09,0.49}{##1}}}
\expandafter\def\csname PY@tok@m\endcsname{\def\PY@tc##1{\textcolor[rgb]{0.40,0.40,0.40}{##1}}}
\expandafter\def\csname PY@tok@mh\endcsname{\def\PY@tc##1{\textcolor[rgb]{0.40,0.40,0.40}{##1}}}
\expandafter\def\csname PY@tok@go\endcsname{\def\PY@tc##1{\textcolor[rgb]{0.53,0.53,0.53}{##1}}}
\expandafter\def\csname PY@tok@ge\endcsname{\let\PY@it=\textit}
\expandafter\def\csname PY@tok@vc\endcsname{\def\PY@tc##1{\textcolor[rgb]{0.10,0.09,0.49}{##1}}}
\expandafter\def\csname PY@tok@il\endcsname{\def\PY@tc##1{\textcolor[rgb]{0.40,0.40,0.40}{##1}}}
\expandafter\def\csname PY@tok@cs\endcsname{\let\PY@it=\textit\def\PY@tc##1{\textcolor[rgb]{0.25,0.50,0.50}{##1}}}
\expandafter\def\csname PY@tok@cp\endcsname{\def\PY@tc##1{\textcolor[rgb]{0.74,0.48,0.00}{##1}}}
\expandafter\def\csname PY@tok@gi\endcsname{\def\PY@tc##1{\textcolor[rgb]{0.00,0.63,0.00}{##1}}}
\expandafter\def\csname PY@tok@gh\endcsname{\let\PY@bf=\textbf\def\PY@tc##1{\textcolor[rgb]{0.00,0.00,0.50}{##1}}}
\expandafter\def\csname PY@tok@ni\endcsname{\let\PY@bf=\textbf\def\PY@tc##1{\textcolor[rgb]{0.60,0.60,0.60}{##1}}}
\expandafter\def\csname PY@tok@nl\endcsname{\def\PY@tc##1{\textcolor[rgb]{0.63,0.63,0.00}{##1}}}
\expandafter\def\csname PY@tok@nn\endcsname{\let\PY@bf=\textbf\def\PY@tc##1{\textcolor[rgb]{0.00,0.00,1.00}{##1}}}
\expandafter\def\csname PY@tok@no\endcsname{\def\PY@tc##1{\textcolor[rgb]{0.53,0.00,0.00}{##1}}}
\expandafter\def\csname PY@tok@na\endcsname{\def\PY@tc##1{\textcolor[rgb]{0.49,0.56,0.16}{##1}}}
\expandafter\def\csname PY@tok@nb\endcsname{\def\PY@tc##1{\textcolor[rgb]{0.00,0.50,0.00}{##1}}}
\expandafter\def\csname PY@tok@nc\endcsname{\let\PY@bf=\textbf\def\PY@tc##1{\textcolor[rgb]{0.00,0.00,1.00}{##1}}}
\expandafter\def\csname PY@tok@nd\endcsname{\def\PY@tc##1{\textcolor[rgb]{0.67,0.13,1.00}{##1}}}
\expandafter\def\csname PY@tok@ne\endcsname{\let\PY@bf=\textbf\def\PY@tc##1{\textcolor[rgb]{0.82,0.25,0.23}{##1}}}
\expandafter\def\csname PY@tok@nf\endcsname{\def\PY@tc##1{\textcolor[rgb]{0.00,0.00,1.00}{##1}}}
\expandafter\def\csname PY@tok@si\endcsname{\let\PY@bf=\textbf\def\PY@tc##1{\textcolor[rgb]{0.73,0.40,0.53}{##1}}}
\expandafter\def\csname PY@tok@s2\endcsname{\def\PY@tc##1{\textcolor[rgb]{0.73,0.13,0.13}{##1}}}
\expandafter\def\csname PY@tok@vi\endcsname{\def\PY@tc##1{\textcolor[rgb]{0.10,0.09,0.49}{##1}}}
\expandafter\def\csname PY@tok@nt\endcsname{\let\PY@bf=\textbf\def\PY@tc##1{\textcolor[rgb]{0.00,0.50,0.00}{##1}}}
\expandafter\def\csname PY@tok@nv\endcsname{\def\PY@tc##1{\textcolor[rgb]{0.10,0.09,0.49}{##1}}}
\expandafter\def\csname PY@tok@s1\endcsname{\def\PY@tc##1{\textcolor[rgb]{0.73,0.13,0.13}{##1}}}
\expandafter\def\csname PY@tok@kd\endcsname{\let\PY@bf=\textbf\def\PY@tc##1{\textcolor[rgb]{0.00,0.50,0.00}{##1}}}
\expandafter\def\csname PY@tok@sh\endcsname{\def\PY@tc##1{\textcolor[rgb]{0.73,0.13,0.13}{##1}}}
\expandafter\def\csname PY@tok@sc\endcsname{\def\PY@tc##1{\textcolor[rgb]{0.73,0.13,0.13}{##1}}}
\expandafter\def\csname PY@tok@sx\endcsname{\def\PY@tc##1{\textcolor[rgb]{0.00,0.50,0.00}{##1}}}
\expandafter\def\csname PY@tok@bp\endcsname{\def\PY@tc##1{\textcolor[rgb]{0.00,0.50,0.00}{##1}}}
\expandafter\def\csname PY@tok@c1\endcsname{\let\PY@it=\textit\def\PY@tc##1{\textcolor[rgb]{0.25,0.50,0.50}{##1}}}
\expandafter\def\csname PY@tok@kc\endcsname{\let\PY@bf=\textbf\def\PY@tc##1{\textcolor[rgb]{0.00,0.50,0.00}{##1}}}
\expandafter\def\csname PY@tok@c\endcsname{\let\PY@it=\textit\def\PY@tc##1{\textcolor[rgb]{0.25,0.50,0.50}{##1}}}
\expandafter\def\csname PY@tok@mf\endcsname{\def\PY@tc##1{\textcolor[rgb]{0.40,0.40,0.40}{##1}}}
\expandafter\def\csname PY@tok@err\endcsname{\def\PY@bc##1{\setlength{\fboxsep}{0pt}\fcolorbox[rgb]{1.00,0.00,0.00}{1,1,1}{\strut ##1}}}
\expandafter\def\csname PY@tok@mb\endcsname{\def\PY@tc##1{\textcolor[rgb]{0.40,0.40,0.40}{##1}}}
\expandafter\def\csname PY@tok@ss\endcsname{\def\PY@tc##1{\textcolor[rgb]{0.10,0.09,0.49}{##1}}}
\expandafter\def\csname PY@tok@sr\endcsname{\def\PY@tc##1{\textcolor[rgb]{0.73,0.40,0.53}{##1}}}
\expandafter\def\csname PY@tok@mo\endcsname{\def\PY@tc##1{\textcolor[rgb]{0.40,0.40,0.40}{##1}}}
\expandafter\def\csname PY@tok@kn\endcsname{\let\PY@bf=\textbf\def\PY@tc##1{\textcolor[rgb]{0.00,0.50,0.00}{##1}}}
\expandafter\def\csname PY@tok@mi\endcsname{\def\PY@tc##1{\textcolor[rgb]{0.40,0.40,0.40}{##1}}}
\expandafter\def\csname PY@tok@gp\endcsname{\let\PY@bf=\textbf\def\PY@tc##1{\textcolor[rgb]{0.00,0.00,0.50}{##1}}}
\expandafter\def\csname PY@tok@o\endcsname{\def\PY@tc##1{\textcolor[rgb]{0.40,0.40,0.40}{##1}}}
\expandafter\def\csname PY@tok@kr\endcsname{\let\PY@bf=\textbf\def\PY@tc##1{\textcolor[rgb]{0.00,0.50,0.00}{##1}}}
\expandafter\def\csname PY@tok@s\endcsname{\def\PY@tc##1{\textcolor[rgb]{0.73,0.13,0.13}{##1}}}
\expandafter\def\csname PY@tok@kp\endcsname{\def\PY@tc##1{\textcolor[rgb]{0.00,0.50,0.00}{##1}}}
\expandafter\def\csname PY@tok@w\endcsname{\def\PY@tc##1{\textcolor[rgb]{0.73,0.73,0.73}{##1}}}
\expandafter\def\csname PY@tok@kt\endcsname{\def\PY@tc##1{\textcolor[rgb]{0.69,0.00,0.25}{##1}}}
\expandafter\def\csname PY@tok@ow\endcsname{\let\PY@bf=\textbf\def\PY@tc##1{\textcolor[rgb]{0.67,0.13,1.00}{##1}}}
\expandafter\def\csname PY@tok@sb\endcsname{\def\PY@tc##1{\textcolor[rgb]{0.73,0.13,0.13}{##1}}}
\expandafter\def\csname PY@tok@k\endcsname{\let\PY@bf=\textbf\def\PY@tc##1{\textcolor[rgb]{0.00,0.50,0.00}{##1}}}
\expandafter\def\csname PY@tok@se\endcsname{\let\PY@bf=\textbf\def\PY@tc##1{\textcolor[rgb]{0.73,0.40,0.13}{##1}}}
\expandafter\def\csname PY@tok@sd\endcsname{\let\PY@it=\textit\def\PY@tc##1{\textcolor[rgb]{0.73,0.13,0.13}{##1}}}

\def\PYZbs{\char`\\}
\def\PYZus{\char`\_}
\def\PYZob{\char`\{}
\def\PYZcb{\char`\}}
\def\PYZca{\char`\^}
\def\PYZam{\char`\&}
\def\PYZlt{\char`\<}
\def\PYZgt{\char`\>}
\def\PYZsh{\char`\#}
\def\PYZpc{\char`\%}
\def\PYZdl{\char`\$}
\def\PYZhy{\char`\-}
\def\PYZsq{\char`\'}
\def\PYZdq{\char`\"}
\def\PYZti{\char`\~}
% for compatibility with earlier versions
\def\PYZat{@}
\def\PYZlb{[}
\def\PYZrb{]}
\makeatother


    % Exact colors from NB
    \definecolor{incolor}{rgb}{0.0, 0.0, 0.5}
    \definecolor{outcolor}{rgb}{0.545, 0.0, 0.0}



    
    % Prevent overflowing lines due to hard-to-break entities
    \sloppy 
    % Setup hyperref package
    \hypersetup{
      breaklinks=true,  % so long urls are correctly broken across lines
      colorlinks=true,
      urlcolor=blue,
      linkcolor=darkorange,
      citecolor=darkgreen,
      }
    % Slightly bigger margins than the latex defaults
    
    \geometry{verbose,tmargin=1in,bmargin=1in,lmargin=1in,rmargin=1in}
    
    

\usepackage{Sweave}
\begin{document}
\Sconcordance{concordance:Summary.tex:Summary.Rnw:%
1 220 1 1 0 700 1}

    
    
    \maketitle
    
    

    
    \section{Summaries of timing tests}\label{summaries-of-timing-tests}

We compare several optimizers, in both
\href{http://www.R-project.org}{R} and
\href{http://julialang.org}{Julia}, fitting a selection of linear mixed
models. In \textbf{R} the optimizers are called by \texttt{lmer} from
the \href{https://github.com/lme4/lme4}{lme4 package} (version 1.1-8).
In \textbf{Julia} the \texttt{lmm} function from the
\href{https://github.com/dmbates/MixedModels.jl}{MixedModels package}
calls the optimizers.

There are differences in the model formulations in \texttt{lme4} and in
\texttt{MixedModels}. The numerical representation of the model in
\texttt{lme4} and the method of evaluating the objective, described in
\href{http://arxiv.org/abs/1406.5823}{this paper}, is the same for all
models. In \texttt{MixedModels} there are specialized representations
for some model forms, such as models with a single grouping factor for
the random effects. Some of the specialized representations allow for
evaluation of the gradient of the objects, which can enhance convergence
(but, interestingly, sometimes can impede convergence).

\subsection{Methodology}\label{methodology}

To provide consistency we have copied all the data sets used in the
timings to the \texttt{Timings} package itself. We have done all timings
on the same computer. This computer has a relatively recent Intel
processor and we used the
\href{https://software.intel.com/en-us/intel-mkl}{Intel Math Kernel
Library (MKL)} with Julia. We attempted to use
\href{www.revolutionanalytics.com/revolution-r-open}{Revolution R Open
(RRO)} as the R implementation as it can be configured with MKL.
However, we ran into version problems with this so we used the standard
Ubuntu version of R linked against OpenBLAS, which is also
multi-threaded.

Variables were renamed in the pattern: - \textbf{Y} the response -
\textbf{A}, \textbf{B}, $\dots$ categorical covariates - \textbf{G},
\textbf{H}, \textbf{I}, $\dots$ grouping factors for random effects -
\textbf{U}, \textbf{V}, $\dots$ (skipping \textbf{Y}) continuous
covariates

The timing results are saved in \href{http://json.org}{JSON (JavaScript
Object Notation)} files in the directory accessible as

\begin{Shaded}
\begin{Highlighting}[]
\KeywordTok{system.file}\NormalTok{(}\StringTok{"JSON"}\NormalTok{,}\DataTypeTok{package=}\StringTok{"Timings"}\NormalTok{)}
\end{Highlighting}
\end{Shaded}

within \textbf{R}. The directory name will end with
\texttt{./Timings/inst/JSON/} in the package source directory, for
example the result of cloning the
\href{https://github.com/Stat990-033/Timings}{github repository}. There
is one \texttt{.json} file for each data set. Each such file contains
results on timings of one or more models.

The \texttt{Timings} package for \textbf{R} provides a \texttt{retime}
function that takes the name of one of these JSON files and, optionally,
the name of a file with the updated timings. Similarly there are some
source files for Julia retimings.

    \begin{Verbatim}[commandchars=\\\{\}]
{\color{incolor}In [{\color{incolor}1}]:} \PY{n}{include}\PY{p}{(}\PY{l+s}{\PYZdq{}}\PY{l+s}{../julia/retime.jl}\PY{l+s}{\PYZdq{}}\PY{p}{)}
        \PY{n}{retime}\PY{p}{(}\PY{l+s}{\PYZdq{}}\PY{l+s}{../JSON/Alfalfa.json}\PY{l+s}{\PYZdq{}}\PY{p}{,}\PY{l+s}{\PYZdq{}}\PY{l+s}{/tmp/Alfalfa.json}\PY{l+s}{\PYZdq{}}\PY{p}{)}\PY{p}{;}
        \PY{n}{retime}\PY{p}{(}\PY{l+s}{\PYZdq{}}\PY{l+s}{../JSON/Alfalfa.json}\PY{l+s}{\PYZdq{}}\PY{p}{,}\PY{l+s}{\PYZdq{}}\PY{l+s}{/tmp/Alfalfa.json}\PY{l+s}{\PYZdq{}}\PY{p}{)}\PY{p}{;}
\end{Verbatim}

    \begin{Verbatim}[commandchars=\\\{\}]
dsname => "Alfalfa"
    \end{Verbatim}

    \begin{Verbatim}[commandchars=\\\{\}]

        `parse` has no method matching parse(::Int64)
    while loading In[1], in expression starting on line 2

        

         in retime at /home/bates/git/Timings/inst/julia/retime.jl:17

    \end{Verbatim}

    The timing was repeated so that compilation time is not included in the
results. This repetition is only needed once per session.

A careful examination of these results shows that the main differences
in the Julia timings (the R timings are merely reported, not evaluated)
are that the \texttt{LN\_BOBYQA} and \texttt{LD\_MMA} optimizers are
much faster in the second run. This is because much of the code needs to
be compiled the first time that a derivative-free optimizer and a
derivative-based optimizer are used.

The names of the optimizers used with \texttt{lmm} are those from the
\href{https://github.com/JuliaOpt/NLopt.jl}{NLopt package} for
\textbf{Julia}. Names that begin with \texttt{LD\_} are gradient-based
methods. Names that begin with \texttt{LN\_} are derivative-free
methods. There is one other derivative-free method, \texttt{LN\_PRAXIS},
available in the \texttt{NLopt} package but, for some reason, it can
hang on very simple problems like this. Frequently we omit it.

The optimizers used with \texttt{lmer} include the \texttt{Nelder\_Mead}
optimizer built into the \texttt{lme4} package, the \texttt{bobyqa}
optimizer from the
\href{http://cran.rstudio.com/web/packages/minqa/index.html}{minqa
package}, the derivative-free optimizers from the
\href{http://cran.rstudio.com/web/packages/nloptr/index.html}{nloptr
package} and several optimizers from the
\href{http://cran.rstudio.com/web/packages/optimx/index.html}{optimx
pacakge}.

The \texttt{optimx:bobyqa} optimizer is just a wrapper around
\texttt{bobyqa} (bounded optimization by quadratic approximation) from
the \texttt{minqa} package and should provide results similar to those
from the \texttt{bobyqa} optimizer. For some reason the number of
function evaluations is not reported for the version in \texttt{optimx}.

The optimizers from \texttt{nloptr} (i.e.~those whose names begin with
\texttt{NLOPT\_LN\_}) use the same underlying code as do the similarly
named optimizers in the \texttt{NLopt} package for \textbf{Julia}. The
number of iterations to convergence should be similar for the same
underlying code, although not nessarily exactly the same because the
evaluation of the objective in \textbf{R} and in \textbf{Julia} may
produce slightly different answers. Also the convergence criteria in the
\textbf{Julia} version are more strict than those in the \textbf{R}
version

Also shown are the value of the criterion (negative twice the
log-likelihood, lower is better) achieved, the elapsed time and the
number of function and gradient evaluations. The \texttt{nopt} value is
the number of parameters in the optimization problem. \texttt{mtype} is
the model type in the \textbf{Julia} code. There are special methods for
solving the penalized least squares (PLS) problem, and for evaluating
the objective and its gradient when there is only one grouping factor
for the random effects. The model type is called \texttt{PLSOne}.

The \texttt{Alfalfa} example is a particularly easy one and all of the
optimizerws converge to an objective value close to -10.81023 in less
than 0.6 seconds.

\subsubsection{Tabulating results}\label{tabulating-results}

For the \texttt{Alfalfa} data there is not much of a burden in refitting
the model with all the \textbf{Julia} optimizers just to get the table
shown above. But other examples can take an hour or more to converge and
we don't really need to refit them every time. The \texttt{tabulate.jl}
file contains a function \texttt{optdir} to create a \texttt{DataFrame}
from the results of all the model fits.

    \begin{Verbatim}[commandchars=\\\{\}]
{\color{incolor}In [{\color{incolor}2}]:} \PY{n}{include}\PY{p}{(}\PY{l+s}{\PYZdq{}}\PY{l+s}{../julia/tabulate.jl}\PY{l+s}{\PYZdq{}}\PY{p}{)}
        \PY{n}{res} \PY{o}{=} \PY{n}{optdir}\PY{p}{(}\PY{l+s}{\PYZdq{}}\PY{l+s}{../JSON}\PY{l+s}{\PYZdq{}}\PY{p}{)}\PY{p}{;}
        \PY{n}{res}\PY{p}{[}\PY{l+m+mi}{1}\PY{p}{:}\PY{l+m+mi}{30}\PY{p}{,}\PY{p}{[}\PY{l+m+mi}{1}\PY{p}{,}\PY{l+m+mi}{2}\PY{p}{,}\PY{l+m+mi}{3}\PY{p}{,}\PY{l+m+mi}{6}\PY{p}{,}\PY{l+m+mi}{7}\PY{p}{,}\PY{l+m+mi}{8}\PY{p}{,}\PY{l+m+mi}{9}\PY{p}{]}\PY{p}{]}
\end{Verbatim}

            \begin{Verbatim}[commandchars=\\\{\}]
{\color{outcolor}Out[{\color{outcolor}2}]:} 30x7 DataFrame
        | Row | opt        | dsname          | n      | np | excess  | time    |
        |-----|------------|-----------------|--------|----|---------|---------|
        | 1   | "LD\_CCSAQ" | "Alfalfa"       | 72     | 1  | 0.0     | 0.0017  |
        | 2   | "LD\_CCSAQ" | "AvgDailyGain"  | 32     | 1  | 0.0     | 0.0014  |
        | 3   | "LD\_CCSAQ" | "AvgDailyGain"  | 32     | 1  | 0.0     | 0.0014  |
        | 4   | "LD\_CCSAQ" | "BIB"           | 24     | 1  | 0.0     | 0.0013  |
        | 5   | "LD\_CCSAQ" | "Bond"          | 21     | 1  | 0.0     | 0.0009  |
        | 6   | "LD\_CCSAQ" | "bs10"          | 1104   | 20 | 0.0     | 1.0958  |
        | 7   | "LD\_CCSAQ" | "bs10"          | 1104   | 8  | 39.9948 | 0.0375  |
        | 8   | "LD\_CCSAQ" | "cake"          | 270    | 1  | 0.0     | 0.0033  |
        | 9   | "LD\_CCSAQ" | "Cultivation"   | 24     | 1  | 0.0     | 0.0009  |
        | 10  | "LD\_CCSAQ" | "Demand"        | 77     | 2  | 3.21928 | 0.0055  |
        | 11  | "LD\_CCSAQ" | "dialectNL"     | 225866 | 6  | 0.0     | 6.9896  |
        ⋮
        | 19  | "LD\_CCSAQ" | "gb12"          | 512    | 8  | 103.176 | 0.0218  |
        | 20  | "LD\_CCSAQ" | "HR"            | 120    | 3  | 0.0     | 0.0089  |
        | 21  | "LD\_CCSAQ" | "Hsb82"         | 7185   | 1  | 192.73  | 0.0102  |
        | 22  | "LD\_CCSAQ" | "IncBlk"        | 24     | 1  | 0.55726 | 0.001   |
        | 23  | "LD\_CCSAQ" | "kb07"          | 1790   | 72 | 8.20739 | 17.4698 |
        | 24  | "LD\_CCSAQ" | "Mississippi"   | 37     | 1  | 0.93471 | 0.0006  |
        | 25  | "LD\_CCSAQ" | "mm0"           | 69588  | 6  | 0.0     | 4.8286  |
        | 26  | "LD\_CCSAQ" | "Oxboys"        | 234    | 3  | 136.788 | 0.0169  |
        | 27  | "LD\_CCSAQ" | "PBIB"          | 60     | 1  | 0.0     | 0.0014  |
        | 28  | "LD\_CCSAQ" | "Penicillin"    | 144    | 2  | 0.0     | 0.0131  |
        | 29  | "LD\_CCSAQ" | "Semiconductor" | 48     | 1  | 0.0     | 0.0012  |
        | 30  | "LD\_CCSAQ" | "SIMS"          | 3691   | 3  | 3.60856 | 0.134   |
        
        | Row | reltime |
        |-----|---------|
        | 1   | 1.1342  |
        | 2   | 0.8207  |
        | 3   | 0.9283  |
        | 4   | 0.7897  |
        | 5   | 1.0265  |
        | 6   | 4.4375  |
        | 7   | 0.555   |
        | 8   | 1.3714  |
        | 9   | 0.9925  |
        | 10  | 0.8079  |
        | 11  | 3.9932  |
        ⋮
        | 19  | 0.5604  |
        | 20  | 1.2196  |
        | 21  | 0.4919  |
        | 22  | 0.6573  |
        | 23  | 4.1242  |
        | 24  | 0.6716  |
        | 25  | 4.4016  |
        | 26  | 0.7092  |
        | 27  | 0.9986  |
        | 28  | 4.9414  |
        | 29  | 1.0461  |
        | 30  | 0.9858  |
\end{Verbatim}
        
    The \texttt{time} column is the time in seconds to converge. The
\texttt{reltime} column is the time relative to the \texttt{LN\_BOBYQA}
optimizer in the \texttt{MixedModels} package for \textbf{Julia}.

    \begin{Verbatim}[commandchars=\\\{\}]
{\color{incolor}In [{\color{incolor}3}]:} \PY{n}{res}\PY{p}{[}\PY{n}{res}\PY{p}{[}\PY{p}{:}\PY{n}{opt}\PY{p}{]} \PY{o}{.}\PY{o}{==} \PY{l+s}{\PYZdq{}}\PY{l+s}{NLOPT\PYZus{}LN\PYZus{}BOBYQA}\PY{l+s}{\PYZdq{}}\PY{p}{,}\PY{p}{[}\PY{l+m+mi}{1}\PY{p}{,}\PY{l+m+mi}{2}\PY{p}{,}\PY{l+m+mi}{3}\PY{p}{,}\PY{l+m+mi}{6}\PY{p}{,}\PY{l+m+mi}{7}\PY{p}{,}\PY{l+m+mi}{8}\PY{p}{,}\PY{l+m+mi}{9}\PY{p}{]}\PY{p}{]}
\end{Verbatim}

            \begin{Verbatim}[commandchars=\\\{\}]
{\color{outcolor}Out[{\color{outcolor}3}]:} 49x7 DataFrame
        | Row | opt               | dsname          | n      | np | excess | time   |
        |-----|-------------------|-----------------|--------|----|--------|--------|
        | 1   | "NLOPT\_LN\_BOBYQA" | "Alfalfa"       | 72     | 1  | 0.0    | 0.042  |
        | 2   | "NLOPT\_LN\_BOBYQA" | "Animal"        | 20     | 2  | 0.0    | 0.023  |
        | 3   | "NLOPT\_LN\_BOBYQA" | "Assay"         | 60     | 2  | 1.0e-5 | 0.032  |
        | 4   | "NLOPT\_LN\_BOBYQA" | "AvgDailyGain"  | 32     | 1  | 0.0    | 0.02   |
        | 5   | "NLOPT\_LN\_BOBYQA" | "AvgDailyGain"  | 32     | 1  | 0.0    | 0.02   |
        | 6   | "NLOPT\_LN\_BOBYQA" | "BIB"           | 24     | 1  | 0.0    | 0.02   |
        | 7   | "NLOPT\_LN\_BOBYQA" | "Bond"          | 21     | 1  | 0.0    | 0.02   |
        | 8   | "NLOPT\_LN\_BOBYQA" | "bs10"          | 1104   | 20 | 1.0e-5 | 4.661  |
        | 9   | "NLOPT\_LN\_BOBYQA" | "bs10"          | 1104   | 8  | 0.0    | 1.057  |
        | 10  | "NLOPT\_LN\_BOBYQA" | "cake"          | 270    | 1  | 0.0    | 0.053  |
        | 11  | "NLOPT\_LN\_BOBYQA" | "Chem97"        | 31022  | 2  | 0.0    | 0.632  |
        ⋮
        | 38  | "NLOPT\_LN\_BOBYQA" | "PBIB"          | 60     | 1  | 0.0    | 0.018  |
        | 39  | "NLOPT\_LN\_BOBYQA" | "Penicillin"    | 144    | 2  | 0.0    | 0.023  |
        | 40  | "NLOPT\_LN\_BOBYQA" | "Poems"         | 275996 | 3  | 0.0    | 21.309 |
        | 41  | "NLOPT\_LN\_BOBYQA" | "ScotsSec"      | 3435   | 2  | 0.0    | 0.076  |
        | 42  | "NLOPT\_LN\_BOBYQA" | "Semi2"         | 72     | 3  | 0.0    | 0.03   |
        | 43  | "NLOPT\_LN\_BOBYQA" | "Semiconductor" | 48     | 1  | 0.0    | 0.019  |
        | 44  | "NLOPT\_LN\_BOBYQA" | "SIMS"          | 3691   | 3  | 0.0    | 0.15   |
        | 45  | "NLOPT\_LN\_BOBYQA" | "sleepstudy"    | 180    | 3  | 0.0    | 0.037  |
        | 46  | "NLOPT\_LN\_BOBYQA" | "sleepstudy"    | 180    | 2  | 0.0    | 0.024  |
        | 47  | "NLOPT\_LN\_BOBYQA" | "TeachingII"    | 96     | 1  | 0.0    | 0.021  |
        | 48  | "NLOPT\_LN\_BOBYQA" | "Weights"       | 399    | 3  | 1.0e-5 | 0.039  |
        | 49  | "NLOPT\_LN\_BOBYQA" | "WWheat"        | 60     | 3  | 0.0    | 0.025  |
        
        | Row | reltime |
        |-----|---------|
        | 1   | 27.5171 |
        | 2   | 13.9236 |
        | 3   | 10.8009 |
        | 4   | 11.7005 |
        | 5   | 13.6969 |
        | 6   | 12.608  |
        | 7   | 23.2699 |
        | 8   | 18.8753 |
        | 9   | 15.6477 |
        | 10  | 22.0827 |
        | 11  | 3.942   |
        ⋮
        | 38  | 13.1358 |
        | 39  | 8.6481  |
        | 40  | 3.7438  |
        | 41  | 5.0422  |
        | 42  | 9.2732  |
        | 43  | 15.9582 |
        | 44  | 1.1034  |
        | 45  | 5.3039  |
        | 46  | 8.6177  |
        | 47  | 15.3763 |
        | 48  | 1.133   |
        | 49  | 2.3154  |
\end{Verbatim}
        
    \subsection{Proportion converged}\label{proportion-converged}

The most important question regarding the optimizers is whether or not
they have converged to the global optimum. We cannot test this directly.
Instead we use a ``crowd-sourced'' criterion based on the minimum
objective achieved by any of the algorithms. The difference between the
objective achieved by a particular algorithm and this minimum is called
the \texttt{excess}. In the summaries \texttt{excess} is rounded to 5
digits after the decimal so the minimum non-zero \texttt{excess} is
$10^{-5}$.

    \begin{Verbatim}[commandchars=\\\{\}]
{\color{incolor}In [{\color{incolor}4}]:} \PY{n}{res}\PY{p}{[}\PY{n}{res}\PY{p}{[}\PY{p}{:}\PY{n}{opt}\PY{p}{]} \PY{o}{.}\PY{o}{==} \PY{l+s}{\PYZdq{}}\PY{l+s}{LN\PYZus{}BOBYQA}\PY{l+s}{\PYZdq{}}\PY{p}{,}\PY{p}{[}\PY{p}{:}\PY{n}{opt}\PY{p}{,}\PY{p}{:}\PY{n}{dsname}\PY{p}{,}\PY{p}{:}\PY{n}{excess}\PY{p}{]}\PY{p}{]}
\end{Verbatim}

            \begin{Verbatim}[commandchars=\\\{\}]
{\color{outcolor}Out[{\color{outcolor}4}]:} 49x3 DataFrame
        | Row | opt         | dsname          | excess |
        |-----|-------------|-----------------|--------|
        | 1   | "LN\_BOBYQA" | "Alfalfa"       | 0.0    |
        | 2   | "LN\_BOBYQA" | "Animal"        | 0.0    |
        | 3   | "LN\_BOBYQA" | "Assay"         | 0.0    |
        | 4   | "LN\_BOBYQA" | "AvgDailyGain"  | 0.0    |
        | 5   | "LN\_BOBYQA" | "AvgDailyGain"  | 0.0    |
        | 6   | "LN\_BOBYQA" | "BIB"           | 0.0    |
        | 7   | "LN\_BOBYQA" | "Bond"          | 0.0    |
        | 8   | "LN\_BOBYQA" | "bs10"          | 1.0e-5 |
        | 9   | "LN\_BOBYQA" | "bs10"          | 0.0    |
        | 10  | "LN\_BOBYQA" | "cake"          | 0.0    |
        | 11  | "LN\_BOBYQA" | "Chem97"        | 0.0    |
        ⋮
        | 38  | "LN\_BOBYQA" | "PBIB"          | 0.0    |
        | 39  | "LN\_BOBYQA" | "Penicillin"    | 0.0    |
        | 40  | "LN\_BOBYQA" | "Poems"         | 0.0    |
        | 41  | "LN\_BOBYQA" | "ScotsSec"      | 0.0    |
        | 42  | "LN\_BOBYQA" | "Semi2"         | 0.0    |
        | 43  | "LN\_BOBYQA" | "Semiconductor" | 0.0    |
        | 44  | "LN\_BOBYQA" | "SIMS"          | 0.0    |
        | 45  | "LN\_BOBYQA" | "sleepstudy"    | 0.0    |
        | 46  | "LN\_BOBYQA" | "sleepstudy"    | 0.0    |
        | 47  | "LN\_BOBYQA" | "TeachingII"    | 0.0    |
        | 48  | "LN\_BOBYQA" | "Weights"       | 0.0    |
        | 49  | "LN\_BOBYQA" | "WWheat"        | 0.0    |
\end{Verbatim}
        
    If we wish to declare ``converged'' or ``not converged'' according to
the excess objective value we must establish a threshold. An absolute
threshold seems reasonable because the objective, negative twice the
log-likelihood, is on a scale where differences in this objective are
compared to a $\chi^2$ random variable. Thus an excess of $10^{-9}$ or
even $10^{-5}$ is negligible.

For each optimizer we can examine which of the data set/model
combinations resulted in an excess greater than a threshold.

    \begin{Verbatim}[commandchars=\\\{\}]
{\color{incolor}In [{\color{incolor}5}]:} \PY{n}{by}\PY{p}{(}\PY{n}{res}\PY{p}{,}\PY{p}{:}\PY{n}{opt}\PY{p}{)} \PY{k}{do} \PY{n}{df}
            \PY{n}{DataFrame}\PY{p}{(}\PY{n}{attempted}\PY{o}{=}\PY{n}{size}\PY{p}{(}\PY{n}{df}\PY{p}{,}\PY{l+m+mi}{1}\PY{p}{)}\PY{p}{,}\PY{n}{failed}\PY{o}{=}\PY{n}{countnz}\PY{p}{(}\PY{n}{df}\PY{p}{[}\PY{p}{:}\PY{n}{excess}\PY{p}{]} \PY{o}{.}\PY{o}{\PYZgt{}} \PY{l+m+mf}{0.02}\PY{p}{)}\PY{p}{)}
        \PY{k}{end}
\end{Verbatim}

            \begin{Verbatim}[commandchars=\\\{\}]
{\color{outcolor}Out[{\color{outcolor}5}]:} 26x3 DataFrame
        | Row | opt                          | attempted | failed |
        |-----|------------------------------|-----------|--------|
        | 1   | "LD\_CCSAQ"                   | 35        | 11     |
        | 2   | "LD\_LBFGS"                   | 35        | 11     |
        | 3   | "LD\_MMA"                     | 36        | 5      |
        | 4   | "LD\_SLSQP"                   | 35        | 4      |
        | 5   | "LD\_TNEWTON"                 | 34        | 8      |
        | 6   | "LD\_TNEWTON\_PRECOND"         | 34        | 8      |
        | 7   | "LD\_TNEWTON\_PRECOND\_RESTART" | 33        | 12     |
        | 8   | "LD\_TNEWTON\_RESTART"         | 34        | 11     |
        | 9   | "LD\_VAR1"                    | 35        | 10     |
        | 10  | "LD\_VAR2"                    | 35        | 10     |
        | 11  | "LN\_BOBYQA"                  | 49        | 0      |
        ⋮
        | 15  | "LN\_SBPLX"                   | 49        | 2      |
        | 16  | "NLOPT\_LN\_BOBYQA"            | 49        | 0      |
        | 17  | "NLOPT\_LN\_COBYLA"            | 48        | 2      |
        | 18  | "NLOPT\_LN\_NELDERMEAD"        | 47        | 6      |
        | 19  | "NLOPT\_LN\_PRAXIS"            | 18        | 4      |
        | 20  | "NLOPT\_LN\_SBPLX"             | 48        | 2      |
        | 21  | "Nelder\_Mead"                | 49        | 8      |
        | 22  | "bobyqa"                     | 49        | 2      |
        | 23  | "optimx:L-BFGS-B"            | 49        | 0      |
        | 24  | "optimx:bobyqa"              | 49        | 2      |
        | 25  | "optimx:nlminb"              | 49        | 0      |
        | 26  | "optimx:spg"                 | 49        | 4      |
\end{Verbatim}
        
    At this threshold the most reliable algorithm in Julia is
\texttt{LN\_BOBYQA}. In R the most reliable algorithms are
\texttt{NLOPT\_LN\_BOBYQA}, \texttt{optimx:L-BFGS-B} and
\texttt{optimx:nlminb}. It is interesting that nlminb is reliable as I
felt that it wasn't converging well when it was the default optimizer in
\texttt{lmer}.

Interestingly, the derviative-based algorithms in \texttt{NLopt} were
not as reliable as the derivative-free algorithms. The most likely
explanation is that I don't have the gradient coded properly.

The Nelder-Mead simplex algorithm did not perform well, failing on 8 out
of 48 cases. For many of these the value at which convergence was
declared was far from the optimum.

    \begin{Verbatim}[commandchars=\\\{\}]
{\color{incolor}In [{\color{incolor}6}]:} \PY{n}{noncvg} \PY{o}{=} 
        \PY{n}{by}\PY{p}{(}\PY{n}{res}\PY{p}{,}\PY{p}{:}\PY{n}{opt}\PY{p}{)} \PY{k}{do} \PY{n}{df}
            \PY{n}{df}\PY{p}{[}\PY{n}{df}\PY{p}{[}\PY{p}{:}\PY{n}{excess}\PY{p}{]} \PY{o}{.}\PY{o}{\PYZgt{}} \PY{l+m+mf}{0.005}\PY{p}{,}\PY{p}{[}\PY{p}{:}\PY{n}{dsname}\PY{p}{,}\PY{p}{:}\PY{n}{excess}\PY{p}{,}\PY{p}{:}\PY{n}{time}\PY{p}{,}\PY{p}{:}\PY{n}{reltime}\PY{p}{,}\PY{p}{:}\PY{n}{np}\PY{p}{,}\PY{p}{:}\PY{n}{n}\PY{p}{]}\PY{p}{]}
        \PY{k}{end}\PY{p}{;}
        \PY{n}{dfselect}\PY{p}{(}\PY{n}{df}\PY{p}{:}\PY{p}{:}\PY{n}{AbstractDataFrame}\PY{p}{,}\PY{n}{col}\PY{p}{:}\PY{p}{:}\PY{n}{Symbol}\PY{p}{,}\PY{n}{val}\PY{p}{)} \PY{o}{=} \PY{n}{df}\PY{p}{[}\PY{n}{df}\PY{p}{[}\PY{n}{col}\PY{p}{]} \PY{o}{.}\PY{o}{==} \PY{n}{val}\PY{p}{,} \PY{p}{:}\PY{p}{]}
        \PY{n}{dfselect}\PY{p}{(}\PY{n}{noncvg}\PY{p}{,}\PY{p}{:}\PY{n}{opt}\PY{p}{,}\PY{l+s}{\PYZdq{}}\PY{l+s}{Nelder\PYZus{}Mead}\PY{l+s}{\PYZdq{}}\PY{p}{)}
\end{Verbatim}

            \begin{Verbatim}[commandchars=\\\{\}]
{\color{outcolor}Out[{\color{outcolor}6}]:} 8x7 DataFrame
        | Row | opt           | dsname        | excess  | time    | reltime | np |
        |-----|---------------|---------------|---------|---------|---------|----|
        | 1   | "Nelder\_Mead" | "bs10"        | 71.3859 | 145.368 | 588.684 | 20 |
        | 2   | "Nelder\_Mead" | "d3"          | 317.59  | 454.519 | 4.2502  | 9  |
        | 3   | "Nelder\_Mead" | "dialectNL"   | 181.632 | 54.541  | 31.1594 | 6  |
        | 4   | "Nelder\_Mead" | "gb12"        | 78.7119 | 38.206  | 189.605 | 20 |
        | 5   | "Nelder\_Mead" | "kb07"        | 398.732 | 2825.46 | 667.015 | 72 |
        | 6   | "Nelder\_Mead" | "kb07"        | 403.478 | 269.436 | 383.087 | 16 |
        | 7   | "Nelder\_Mead" | "Mississippi" | 0.04272 | 0.018   | 20.3989 | 1  |
        | 8   | "Nelder\_Mead" | "mm0"         | 181.632 | 76.87   | 70.0716 | 6  |
        
        | Row | n      |
        |-----|--------|
        | 1   | 1104   |
        | 2   | 130418 |
        | 3   | 225866 |
        | 4   | 512    |
        | 5   | 1790   |
        | 6   | 1790   |
        | 7   | 37     |
        | 8   | 69588  |
\end{Verbatim}
        
    \begin{Verbatim}[commandchars=\\\{\}]
{\color{incolor}In [{\color{incolor}7}]:} \PY{n}{dfselect}\PY{p}{(}\PY{n}{noncvg}\PY{p}{,}\PY{p}{:}\PY{n}{opt}\PY{p}{,}\PY{l+s}{\PYZdq{}}\PY{l+s}{NLOPT\PYZus{}LN\PYZus{}NELDERMEAD}\PY{l+s}{\PYZdq{}}\PY{p}{)}
\end{Verbatim}

            \begin{Verbatim}[commandchars=\\\{\}]
{\color{outcolor}Out[{\color{outcolor}7}]:} 6x7 DataFrame
        | Row | opt                   | dsname  | excess  | time    | reltime | np |
        |-----|-----------------------|---------|---------|---------|---------|----|
        | 1   | "NLOPT\_LN\_NELDERMEAD" | "Assay" | 0.05942 | 0.042   | 14.1762 | 2  |
        | 2   | "NLOPT\_LN\_NELDERMEAD" | "bs10"  | 0.97096 | 88.286  | 357.524 | 20 |
        | 3   | "NLOPT\_LN\_NELDERMEAD" | "d3"    | 100.191 | 627.931 | 5.8718  | 9  |
        | 4   | "NLOPT\_LN\_NELDERMEAD" | "gb12"  | 0.83883 | 52.411  | 260.1   | 20 |
        | 5   | "NLOPT\_LN\_NELDERMEAD" | "kb07"  | 88.7686 | 2839.07 | 670.229 | 72 |
        | 6   | "NLOPT\_LN\_NELDERMEAD" | "kb07"  | 3.49868 | 198.845 | 282.72  | 16 |
        
        | Row | n      |
        |-----|--------|
        | 1   | 60     |
        | 2   | 1104   |
        | 3   | 130418 |
        | 4   | 512    |
        | 5   | 1790   |
        | 6   | 1790   |
\end{Verbatim}
        
    The \texttt{Nelder\_Mead} algorithm, either in the native form in
\texttt{lmer} or in the \texttt{NLopt} implementation performed poorly
on those cases with many parameters to optimize. It was both unreliable
and slow, taking over 45 minutes to reach a spurious optimum on the
``maximal'' model (in the sense of Barr et al., 2012) for the
\texttt{kb07} data from Kronmueller and Barr (2007). This is not
terribly surprising given that the model is horribly overparameterized,
but still it shows that this algorithm is not a good choice in these
cases.

We note in passing that all the models involving fitting 20 or more
parameters are ``maximal'' models in the sense of Barr et al., 2012.
Such models can present difficult optimization problems because they are
severely overparameterized and inevitably converge on the boundary of
the allowable parameter space. Whether or not it is sensible to compare
results on such extreme cases is not clear.

The \texttt{SBPLX} (subplex) algorithm, which is an enhancement of
\texttt{Nelder\_Mead}, does better in these cases but is still rather
slow.

    \begin{Verbatim}[commandchars=\\\{\}]
{\color{incolor}In [{\color{incolor}8}]:} \PY{n}{dfselect}\PY{p}{(}\PY{n}{noncvg}\PY{p}{,}\PY{p}{:}\PY{n}{opt}\PY{p}{,}\PY{l+s}{\PYZdq{}}\PY{l+s}{NLOPT\PYZus{}LN\PYZus{}SBPLX}\PY{l+s}{\PYZdq{}}\PY{p}{)}
\end{Verbatim}

            \begin{Verbatim}[commandchars=\\\{\}]
{\color{outcolor}Out[{\color{outcolor}8}]:} 2x7 DataFrame
        | Row | opt              | dsname | excess  | time    | reltime | np | n    |
        |-----|------------------|--------|---------|---------|---------|----|------|
        | 1   | "NLOPT\_LN\_SBPLX" | "gb12" | 0.82219 | 3.813   | 18.9228 | 20 | 512  |
        | 2   | "NLOPT\_LN\_SBPLX" | "kb07" | 4.96546 | 564.688 | 133.308 | 72 | 1790 |
\end{Verbatim}
        
    By comparison, the \texttt{LN\_BOBYQA} algorithm converges quite rapidly
on the \texttt{kb07} models.

    \begin{Verbatim}[commandchars=\\\{\}]
{\color{incolor}In [{\color{incolor}9}]:} \PY{n}{bobyqa} \PY{o}{=} \PY{n}{res}\PY{p}{[}\PY{n+nb}{convert}\PY{p}{(}\PY{n}{Array}\PY{p}{,}\PY{n}{res}\PY{p}{[}\PY{p}{:}\PY{n}{opt}\PY{p}{]} \PY{o}{.}\PY{o}{==} \PY{l+s}{\PYZdq{}}\PY{l+s}{LN\PYZus{}BOBYQA}\PY{l+s}{\PYZdq{}}\PY{p}{)} \PY{o}{\PYZam{}} 
            \PY{n+nb}{convert}\PY{p}{(}\PY{n}{Array}\PY{p}{,}\PY{n}{res}\PY{p}{[}\PY{p}{:}\PY{n}{dsname}\PY{p}{]} \PY{o}{.}\PY{o}{==} \PY{l+s}{\PYZdq{}}\PY{l+s}{kb07}\PY{l+s}{\PYZdq{}}\PY{p}{)}\PY{p}{,}
            \PY{p}{[}\PY{p}{:}\PY{n}{dsname}\PY{p}{,}\PY{p}{:}\PY{n}{excess}\PY{p}{,}\PY{p}{:}\PY{n}{objective}\PY{p}{,}\PY{p}{:}\PY{n}{time}\PY{p}{,}\PY{p}{:}\PY{n}{np}\PY{p}{]}\PY{p}{]}
\end{Verbatim}

            \begin{Verbatim}[commandchars=\\\{\}]
{\color{outcolor}Out[{\color{outcolor}9}]:} 2x5 DataFrame
        | Row | dsname | excess  | objective | time   | np |
        |-----|--------|---------|-----------|--------|----|
        | 1   | "kb07" | 0.01695 | 28586.3   | 4.236  | 72 |
        | 2   | "kb07" | 0.0     | 28670.9   | 0.7033 | 16 |
\end{Verbatim}
        
    \subsection{Relative speed}\label{relative-speed}

We plot the time to convergence, relative to \texttt{LN\_BOBYQA} and on
a logarithmic scale, for each algorithm.

    \begin{Verbatim}[commandchars=\\\{\}]
{\color{incolor}In [{\color{incolor}10}]:} \PY{k}{using} \PY{n}{Gadfly}
         \PY{n}{set\PYZus{}default\PYZus{}plot\PYZus{}size}\PY{p}{(}\PY{l+m+mi}{16}\PY{n}{cm}\PY{p}{,}\PY{l+m+mi}{12}\PY{n}{cm}\PY{p}{)}
         \PY{n}{res}\PY{p}{[}\PY{p}{:}\PY{n}{cvg}\PY{p}{]} \PY{o}{=} \PY{n}{compact}\PY{p}{(}\PY{n}{pool}\PY{p}{(}\PY{p}{[}\PY{n}{e} \PY{o}{\PYZgt{}} \PY{l+m+mf}{0.02} \PY{o}{?} \PY{l+s}{\PYZdq{}}\PY{l+s}{N}\PY{l+s}{\PYZdq{}} \PY{p}{:} \PY{l+s}{\PYZdq{}}\PY{l+s}{Y}\PY{l+s}{\PYZdq{}} \PY{k}{for} \PY{n}{e} \PY{k}{in} \PY{n}{res}\PY{p}{[}\PY{p}{:}\PY{n}{excess}\PY{p}{]}\PY{p}{]}\PY{p}{)}\PY{p}{)}
         \PY{n}{plot}\PY{p}{(}\PY{n}{res}\PY{p}{,}\PY{n}{x}\PY{o}{=}\PY{l+s}{\PYZdq{}}\PY{l+s}{reltime}\PY{l+s}{\PYZdq{}}\PY{p}{,}\PY{n}{y}\PY{o}{=}\PY{l+s}{\PYZdq{}}\PY{l+s}{opt}\PY{l+s}{\PYZdq{}}\PY{p}{,}\PY{n}{color}\PY{o}{=}\PY{l+s}{\PYZdq{}}\PY{l+s}{cvg}\PY{l+s}{\PYZdq{}}\PY{p}{,}
             \PY{n}{Geom}\PY{o}{.}\PY{n}{point}\PY{p}{,}\PY{n}{Scale}\PY{o}{.}\PY{n}{y\PYZus{}discrete}\PY{p}{,}\PY{n}{Scale}\PY{o}{.}\PY{n}{x\PYZus{}log2}\PY{p}{,}
             \PY{n}{Guide}\PY{o}{.}\PY{n}{ylabel}\PY{p}{(}\PY{n}{nothing}\PY{p}{)}\PY{p}{,}
             \PY{n}{Guide}\PY{o}{.}\PY{n}{xlabel}\PY{p}{(}\PY{l+s}{\PYZdq{}}\PY{l+s}{Time to convergence relative to LN\PYZus{}BOBYQA}\PY{l+s}{\PYZdq{}}\PY{p}{)}\PY{p}{)}
\end{Verbatim}
\texttt{\color{outcolor}Out[{\color{outcolor}10}]:}
    
    \begin{center}
    \adjustimage{max size={0.9\linewidth}{0.9\paperheight}}{Summaries_files/Summaries_18_0.pdf}
    \end{center}
    { \hspace*{\fill} \\}
    

    Many of the cases where \texttt{LN\_BOBYQA} is slower than other
algorithms are simple problems that converge in less than 1/5 of a
second for most algorithms.

We will declare a data set to be non-simple if at least one of the
models fit to the data took more than 0.2 seconds to convergence with
\texttt{LN\_BOBYQA}.

    \begin{Verbatim}[commandchars=\\\{\}]
{\color{incolor}In [{\color{incolor}12}]:} \PY{n}{ln\PYZus{}bobyqa} \PY{o}{=} \PY{n}{dfselect}\PY{p}{(}\PY{n}{res}\PY{p}{,}\PY{p}{:}\PY{n}{opt}\PY{p}{,}\PY{l+s}{\PYZdq{}}\PY{l+s}{LN\PYZus{}BOBYQA}\PY{l+s}{\PYZdq{}}\PY{p}{)}\PY{p}{;}
         \PY{n}{nonsimple} \PY{o}{=} \PY{n}{ln\PYZus{}bobyqa}\PY{p}{[}\PY{n}{ln\PYZus{}bobyqa}\PY{p}{[}\PY{p}{:}\PY{n}{time}\PY{p}{]} \PY{o}{.}\PY{o}{\PYZgt{}} \PY{l+m+mf}{0.2}\PY{p}{,}\PY{p}{[}\PY{p}{:}\PY{n}{dsname}\PY{p}{,}\PY{p}{:}\PY{n}{time}\PY{p}{,}\PY{p}{:}\PY{n}{n}\PY{p}{,}\PY{p}{:}\PY{n}{np}\PY{p}{,}\PY{p}{:}\PY{n}{models}\PY{p}{]}\PY{p}{]}
\end{Verbatim}

            \begin{Verbatim}[commandchars=\\\{\}]
{\color{outcolor}Out[{\color{outcolor}12}]:} 10x5 DataFrame
         | Row | dsname      | time   | n      | np |
         |-----|-------------|--------|--------|----|
         | 1   | "bs10"      | 0.2469 | 1104   | 20 |
         | 2   | "d3"        | 106.94 | 130418 | 9  |
         | 3   | "dialectNL" | 1.7504 | 225866 | 6  |
         | 4   | "gb12"      | 0.2015 | 512    | 20 |
         | 5   | "InstEval"  | 2.3539 | 73421  | 2  |
         | 6   | "InstEval"  | 4.3593 | 73421  | 3  |
         | 7   | "kb07"      | 4.236  | 1790   | 72 |
         | 8   | "kb07"      | 0.7033 | 1790   | 16 |
         | 9   | "mm0"       | 1.097  | 69588  | 6  |
         | 10  | "Poems"     | 5.6918 | 275996 | 3  |
         
         | Row | models                                                                                                                                                |
         |-----|-------------------------------------------------------------------------------------------------------------------------------------------------------|
         | 1   | "29"                                                                                                                                                  |
         | 2   | "Y \textasciitilde{} U + (U | G)  + (U | H) + (U | I)"                                                                                                                |
         | 3   | "Y \textasciitilde{} U + V +W +X + Z+A + T +(1 | G) +(0 + V + W+ X | G) +(1 + Z + A | H) +(1 | I)"                                                                    |
         | 4   | "Y \textasciitilde{} 1+S+T+U+V+W+X+Z + (1+S+U+W|G) + (1+S+T+V|H)"                                                                                                     |
         | 5   | "Y \textasciitilde{} 1 + I*A + (1|G) + (1|H)"                                                                                                                         |
         | 6   | "Y \textasciitilde{} 1 + A + (1|G) + (1|H) + (1|I)"                                                                                                                   |
         | 7   | "Y \textasciitilde{} 1+S+T+U+V+W+X+Z + (1+S+T+U+V+W+X+Z|G) + (1+S+T+U+V+W+X+Z|H)"                                                                                     |
         | 8   | "Y \textasciitilde{} 1+S+T+U+V+W+X+Z + (1|G)+(0+S|G)+(0+T|G)+(0+U|G)+(0+V|G)+(0+W|G)+(0+X|G)+(0+Z|G) + (1|H)+(0+S|H)+(0+T|H)+(0+U|H)+(0+V|H)+(0+W|H)+(0+X|H)+(0+Z|H)" |
         | 9   | "Y\textasciitilde{}A*U+(1+U|G)+(1+U|H)"                                                                                                                               |
         | 10  | "Y \textasciitilde{} 1 + U + V + (1|G) + (1|H) + (1|I)"                                                                                                               |
\end{Verbatim}
        
    Notice that these are cases with a large number of observations
(\texttt{n}) or a large number of parameters in the optimization problem
(\texttt{np}) or both.

By comparison, the cases where other algorithms are faster than
\texttt{LN\_BOBYQA} are, for the most part, models and data sets with
few observations and few parameters to optimize. In these circumstances
almost all the optimizers are fast.

    \begin{Verbatim}[commandchars=\\\{\}]
{\color{incolor}In [{\color{incolor}13}]:} \PY{n}{by}\PY{p}{(}\PY{n}{res}\PY{p}{,}\PY{p}{:}\PY{n}{opt}\PY{p}{)} \PY{k}{do} \PY{n}{df}
             \PY{n}{cvgfast} \PY{o}{=} \PY{p}{(}\PY{n}{df}\PY{p}{[}\PY{p}{:}\PY{n}{reltime}\PY{p}{]} \PY{o}{.}\PY{o}{\PYZlt{}} \PY{l+m+mi}{1}\PY{p}{)} \PY{o}{\PYZam{}} \PY{n+nb}{convert}\PY{p}{(}\PY{n}{Array}\PY{p}{,}\PY{n}{df}\PY{p}{[}\PY{p}{:}\PY{n}{cvg}\PY{p}{]} \PY{o}{.}\PY{o}{==} \PY{l+s}{\PYZdq{}}\PY{l+s}{Y}\PY{l+s}{\PYZdq{}}\PY{p}{)}
             \PY{k}{if} \PY{n}{any}\PY{p}{(}\PY{n}{cvgfast}\PY{p}{)}
                 \PY{n}{print}\PY{p}{(}\PY{n}{df}\PY{p}{[}\PY{n}{cvgfast}\PY{p}{,}\PY{p}{[}\PY{p}{:}\PY{n}{opt}\PY{p}{,}\PY{p}{:}\PY{n}{dsname}\PY{p}{,}\PY{p}{:}\PY{n}{time}\PY{p}{,}\PY{p}{:}\PY{n}{np}\PY{p}{,}\PY{p}{:}\PY{n}{n}\PY{p}{]}\PY{p}{]}\PY{p}{)}
                 \PY{n}{println}\PY{p}{(}\PY{p}{)}
             \PY{k}{end}
         \PY{k}{end}\PY{p}{;}
\end{Verbatim}

    \begin{Verbatim}[commandchars=\\\{\}]
8x5 DataFrame
| Row | opt        | dsname         | time   | np | n  |
|-----|------------|----------------|--------|----|----|
| 1   | "LD\_CCSAQ" | "AvgDailyGain" | 0.0014 | 1  | 32 |
| 2   | "LD\_CCSAQ" | "AvgDailyGain" | 0.0014 | 1  | 32 |
| 3   | "LD\_CCSAQ" | "BIB"          | 0.0013 | 1  | 24 |
| 4   | "LD\_CCSAQ" | "Cultivation"  | 0.0009 | 1  | 24 |
| 5   | "LD\_CCSAQ" | "Dyestuff2"    | 0.0005 | 1  | 30 |
| 6   | "LD\_CCSAQ" | "Gasoline"     | 0.0011 | 1  | 32 |
| 7   | "LD\_CCSAQ" | "PBIB"         | 0.0014 | 1  | 60 |
| 8   | "LD\_CCSAQ" | "TeachingII"   | 0.0013 | 1  | 96 |
2x5 DataFrame
| Row | opt        | dsname | time   | np | n   |
|-----|------------|--------|--------|----|-----|
| 1   | "LD\_LBFGS" | "gb12" | 0.1871 | 20 | 512 |
| 2   | "LD\_LBFGS" | "HR"   | 0.004  | 3  | 120 |
12x5 DataFrame
| Row | opt      | dsname         | time   | np | n   |
|-----|----------|----------------|--------|----|-----|
| 1   | "LD\_MMA" | "AvgDailyGain" | 0.0015 | 1  | 32  |
| 2   | "LD\_MMA" | "BIB"          | 0.0012 | 1  | 24  |
| 3   | "LD\_MMA" | "Bond"         | 0.0008 | 1  | 21  |
| 4   | "LD\_MMA" | "Dyestuff2"    | 0.0005 | 1  | 30  |
| 5   | "LD\_MMA" | "Dyestuff"     | 0.0007 | 1  | 30  |
| 6   | "LD\_MMA" | "ergoStool"    | 0.0009 | 1  | 36  |
| 7   | "LD\_MMA" | "Gasoline"     | 0.0012 | 1  | 32  |
| 8   | "LD\_MMA" | "Oxboys"       | 0.0092 | 3  | 234 |
| 9   | "LD\_MMA" | "PBIB"         | 0.0013 | 1  | 60  |
| 10  | "LD\_MMA" | "TeachingII"   | 0.0012 | 1  | 96  |
| 11  | "LD\_MMA" | "Weights"      | 0.027  | 3  | 399 |
| 12  | "LD\_MMA" | "WWheat"       | 0.0084 | 3  | 60  |
20x5 DataFrame
| Row | opt        | dsname          | time   | np | n    |
|-----|------------|-----------------|--------|----|------|
| 1   | "LD\_SLSQP" | "Alfalfa"       | 0.0014 | 1  | 72   |
| 2   | "LD\_SLSQP" | "AvgDailyGain"  | 0.0016 | 1  | 32   |
| 3   | "LD\_SLSQP" | "BIB"           | 0.0011 | 1  | 24   |
| 4   | "LD\_SLSQP" | "Bond"          | 0.0007 | 1  | 21   |
| 5   | "LD\_SLSQP" | "Cultivation"   | 0.0008 | 1  | 24   |
| 6   | "LD\_SLSQP" | "Demand"        | 0.0054 | 2  | 77   |
| 7   | "LD\_SLSQP" | "Dyestuff2"     | 0.0007 | 1  | 30   |
| 8   | "LD\_SLSQP" | "Dyestuff"      | 0.0007 | 1  | 30   |
| 9   | "LD\_SLSQP" | "ergoStool"     | 0.001  | 1  | 36   |
| 10  | "LD\_SLSQP" | "Gasoline"      | 0.0011 | 1  | 32   |
| 11  | "LD\_SLSQP" | "HR"            | 0.0056 | 3  | 120  |
| 12  | "LD\_SLSQP" | "IncBlk"        | 0.0013 | 1  | 24   |
| 13  | "LD\_SLSQP" | "Oxboys"        | 0.0072 | 3  | 234  |
| 14  | "LD\_SLSQP" | "PBIB"          | 0.0009 | 1  | 60   |
| 15  | "LD\_SLSQP" | "Semiconductor" | 0.001  | 1  | 48   |
| 16  | "LD\_SLSQP" | "SIMS"          | 0.0742 | 3  | 3691 |
| 17  | "LD\_SLSQP" | "sleepstudy"    | 0.0052 | 3  | 180  |
| 18  | "LD\_SLSQP" | "TeachingII"    | 0.0011 | 1  | 96   |
| 19  | "LD\_SLSQP" | "Weights"       | 0.0234 | 3  | 399  |
| 20  | "LD\_SLSQP" | "WWheat"        | 0.0079 | 3  | 60   |
3x5 DataFrame
| Row | opt          | dsname    | time   | np | n   |
|-----|--------------|-----------|--------|----|-----|
| 1   | "LD\_TNEWTON" | "HR"      | 0.0059 | 3  | 120 |
| 2   | "LD\_TNEWTON" | "Oxboys"  | 0.0087 | 3  | 234 |
| 3   | "LD\_TNEWTON" | "Weights" | 0.0213 | 3  | 399 |
3x5 DataFrame
| Row | opt                  | dsname    | time   | np | n   |
|-----|----------------------|-----------|--------|----|-----|
| 1   | "LD\_TNEWTON\_PRECOND" | "HR"      | 0.0048 | 3  | 120 |
| 2   | "LD\_TNEWTON\_PRECOND" | "Oxboys"  | 0.0092 | 3  | 234 |
| 3   | "LD\_TNEWTON\_PRECOND" | "Weights" | 0.0152 | 3  | 399 |
1x5 DataFrame
| Row | opt                          | dsname | time  | np | n   |
|-----|------------------------------|--------|-------|----|-----|
| 1   | "LD\_TNEWTON\_PRECOND\_RESTART" | "HR"   | 0.005 | 3  | 120 |
1x5 DataFrame
| Row | opt                  | dsname | time   | np | n   |
|-----|----------------------|--------|--------|----|-----|
| 1   | "LD\_TNEWTON\_RESTART" | "HR"   | 0.0052 | 3  | 120 |
1x5 DataFrame
| Row | opt       | dsname | time   | np | n   |
|-----|-----------|--------|--------|----|-----|
| 1   | "LD\_VAR1" | "HR"   | 0.0044 | 3  | 120 |
1x5 DataFrame
| Row | opt       | dsname | time   | np | n   |
|-----|-----------|--------|--------|----|-----|
| 1   | "LD\_VAR2" | "HR"   | 0.0044 | 3  | 120 |
2x5 DataFrame
| Row | opt         | dsname  | time   | np | n    |
|-----|-------------|---------|--------|----|------|
| 1   | "LN\_COBYLA" | "BIB"   | 0.0014 | 1  | 24   |
| 2   | "LN\_COBYLA" | "Hsb82" | 0.018  | 1  | 7185 |
3x5 DataFrame
| Row | opt             | dsname      | time   | np | n   |
|-----|-----------------|-------------|--------|----|-----|
| 1   | "LN\_NELDERMEAD" | "BIB"       | 0.0015 | 1  | 24  |
| 2   | "LN\_NELDERMEAD" | "Dyestuff2" | 0.0005 | 1  | 30  |
| 3   | "LN\_NELDERMEAD" | "Oxboys"    | 0.0192 | 3  | 234 |
1x5 DataFrame
| Row | opt         | dsname  | time   | np | n      |
|-----|-------------|---------|--------|----|--------|
| 1   | "LN\_PRAXIS" | "Poems" | 5.3089 | 3  | 275996 |
4x5 DataFrame
| Row | opt        | dsname      | time    | np | n      |
|-----|------------|-------------|---------|----|--------|
| 1   | "LN\_SBPLX" | "d3"        | 75.8816 | 9  | 130418 |
| 2   | "LN\_SBPLX" | "Demand"    | 0.0049  | 2  | 77     |
| 3   | "LN\_SBPLX" | "dialectNL" | 1.4855  | 6  | 225866 |
| 4   | "LN\_SBPLX" | "Dyestuff2" | 0.0006  | 1  | 30     |
1x5 DataFrame
| Row | opt               | dsname | time  | np | n    |
|-----|-------------------|--------|-------|----|------|
| 1   | "NLOPT\_LN\_COBYLA" | "SIMS" | 0.092 | 3  | 3691 |
    \end{Verbatim}


    % Add a bibliography block to the postdoc
    
    
    
    \end{document}
